\documentclass[a4paper, 12pt]{article}

\usepackage[english,russian]{babel}
\usepackage[T2A]{fontenc}
\usepackage[utf8]{inputenc}
\usepackage{geometry}
\usepackage{enumitem}
\usepackage{setspace}
\usepackage{amssymb}
\usepackage{graphicx}
\usepackage{wrapfig}
\usepackage{float}
\usepackage{amsmath}
\usepackage{textcomp}
\usepackage{dsfont}

\geometry{top=5mm, left=1cm}
%\setlength{\parindent}{0}
\renewcommand{\arraystretch}{1.2}
\linespread{1}

\begin{document}
    \begin{center}
        \textbf{Сферическая геометрия №3}\\
        Расстояние между точками, углы между прямыми, сферические окружности.
    \end{center}

    \textbf{№ 1}
    Покажите, что сумма смежных углов между сферическими прямыми равна $180^\circ$\\


    \textbf{№ 2}
    Покажите, что вертикальные углы между сферическими прямыми равны\\


    \textbf{№ 3}
    Радиус сферы равен $R$, евклидово расстояние между двумя точками сферы равно $h$,
    чему равно сферическое расстояние между этими точками.\\


    \textbf{№ 4}
    Угол между двумя сферическими прямыми равен $\frac{\pi}{4}$, радиус сферы равен $7$ см.
    Из центра сферы в плоскостях сечений восстановили перпендикуляры так, что получилось 4 точки пересечения со сферой.
    Найдите сферическое расстояние между всеми этими точками.\\


    \textbf{№ 5}
    На сфере радиуса $R$ построена сферическая окружность сферического радиуса $r$.
    Найдите евклидов радиус этой окружности.\\


    \textbf{№ 6}
    Назовем образующей плоскостью большой окружности на сфере плоскость,
    если в результате сечения сферы этой плоскостью получается соответсвующая большая окружность.
    Аналогично назовем образующую плоскость малой окружности на сфере плоскость,
    если в результате сечения сферы этой плоскостью получается соответсвующая малая окружность.

    Большая и малая окружности на одной сфере имеют одну общую точку.
    Чему равен угол между образующими их плоскостями, если сферический радиус малой окружности равен $r$,
    а радиус сферы равен $R$?\\


    \textbf{№ 7}
    Чему равно сферическое расстояние между полярно сопряженными точками, если радиус сферы равен $R$?\\


    \textbf{№ 8}
    Проведены две сферические прямые, пересекающиеся под углом $\alpha$,
    перпендикулярно к ним проведена третья сферическая прямая.
    Чему равен радиус сферы, если сферическое расстояние между точками пересечения
    двух прямых третьей равно $h$?\\


    \textbf{№ 9}
    Может ли сферическая окружность быть сферической прямой?\\


    \textbf{№ 10}
    Сколько существует перпендикуляров к данной прямой, проведенных через точку,
    не лежащую на данной прямой.\\


    \textbf{№ 11}
    Пусть дана прямая, точка на ней и число $d < \pi R$, $R$ - радиус сферы.
    Сколько существует точек на этой прямой, удаленных от данной на сферическое расстояние $d$?
    А если $d = \piR$?\\


    \textbf{№ 12}
    Город A расположен на северном полюсе, а города B и C на экваторе,
    траектории авиарейсов из A в B и из A в C взаимно перпендикулярны.
    Оцените расстояние между B и C, если радиус земли $R = 6400$км.
\end{document}
