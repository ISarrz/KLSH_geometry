\documentclass[a4paper, 12pt]{article}

\usepackage[english,russian]{babel}
\usepackage[T2A]{fontenc}
\usepackage[utf8]{inputenc}
\usepackage{geometry}
\usepackage{enumitem}
\usepackage{setspace}
\usepackage{amssymb}
\usepackage{graphicx}
\usepackage{wrapfig}
\usepackage{float}
\usepackage{amsmath}
\usepackage{textcomp}
\usepackage{dsfont}

\geometry{top=5mm, left=1cm}
%\setlength{\parindent}{0}
\renewcommand{\arraystretch}{1.2}
\linespread{1}

\begin{document}
    \begin{center}
        \textbf{Сферическая геометрия №6}\\
        Теорема косинусов, теорема синусов.
    \end{center}


    \textbf{№ 1}
    На сфере радиуса $R$ дан треугольник с сторонами $a, b, c$ найдите его площадь.\\


    \textbf{№ 2}
    На сфере радиуса $R$ дан треугольник со сторонами $a, b, c$.
    Угол между $a$ и $b$ равен $C$.
    Найдите чему равна сторона $c$, если радиус сферы $R$.\\


    \textbf{№ 3}
    Через боковые стороны равнобедренного сферического треугольника провели среднюю линию.
    Найдите длину средней линии, если известны стороны исходного треугольника и угол противолежащий основанию.\\


    \textbf{№ 4}
    Через боковые стороны равнобедренного сферического треугольника провели среднюю линию.
    Найдите углы получившегося треугольника, если известны стороны исходного треугольника и угол противолежащий основанию.\\


    \textbf{№ 5}
    В сферическом треугольнике известны три стороны и один угол.
    Найдите остальные углы.\\


    \textbf{№ 6}
    В сферическом треугольнике $ABC$ угол $B$ прямой.
    Найдите косинусы, синусы, тангенсы, котангенсы двух других углов,
    если известны стороны треугольника и радиус сферы.\\


    \textbf{№ 7}
    В сферическом треугольнике $ABC$ угол $B$ прямой.
    Найдите сторону $AC$, если известны все углы, сторона $AB$ и радиус сферы.


\end{document}
