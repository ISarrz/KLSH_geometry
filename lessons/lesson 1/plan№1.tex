\documentclass[a4paper, 12pt]{article}

\usepackage[english,russian]{babel}
\usepackage[T2A]{fontenc}
\usepackage[utf8]{inputenc}
\usepackage{geometry}
\usepackage{enumitem}
\usepackage{setspace}
\usepackage{amssymb}
\usepackage{graphicx}
\usepackage{wrapfig}
\usepackage{float}
\usepackage{amsmath}
\usepackage{textcomp}
\usepackage{dsfont}

\geometry{top=5mm, left=1cm}
%\setlength{\parindent}{0}
\renewcommand{\arraystretch}{1.2}
\linespread{1}

\begin{document}
    \begin{center}
        \textbf{Сферическая геометрия план занятия №1}\\
    \end{center}

    Содержание по слайдам:
    \begin{enumerate}
        \item Узнать ожидания от курса,
        установить уровень подготовки, углубленность знаний в области евклидовой геометрии, тригонометрии.
        Узнать мотивацию поступления на курс.

        \item Необходимо наводящими вопросами дать понять, что следует различать определение
        геометрии как науки и как математического объекта.
        Возможны ли разные геометрии?
        Чем планиметрия отличается от стереометрии, бывают ли геометрии больших или меньших размерностей.

        \item Разделение идеи термина геометрия.

        \item Древнейший способ построения геометрии - аксиоматический.
        В III века до нашей эры Евклид в книге «Начала» описал фундаментальные предположения(аксиомы)
        и логическим путем выводил из них теоремы.\\
        Спросить, какие аксиомы школьники знают из евклидовой геометрии.

        \item Ввести понятие функции расстояния.
        Спросить, что по мнению школьников обозначает термин изометрия, выяснить очевидность
        пятого постулата.

        \item Вспоминаем/узнаем аксиомы стереометрии.

        \item Идея этого слайда в том, чтобы думать о геометрическом пространстве как о множестве точек,
        а не наборе элементарных фигур.
        Все фигуры в геометрии можно представлять как подмножества пространства.

        \item Объяснить, что точки в евклидовом пространстве можно задавать координатами,
        декартово произведение.
        Возможно сравнение мощности некоторых множеств, например отрезка с квадратом и так далее.

        \item Смысл этого слайда в подводке к биективному отображению,
        как можно сравнивать два пространства, по каким критериям?

        \item Скорее всего тут придется рассказывать что такое биекция,
        показать, что при изометрии отрезок переходит в отрезок.

        \item Тут необходимо порисовать различные преобразования пространства.
        Показать, что поворот на угол есть последовательное отражение пространства относительно двух
        пересекающихся прямых, параллельный перенос отражение относительно двух параллельных прямых.
        Поговорить о композиции преобразований, как ее понимать

        \item Тут представлены изометрии, вникать в них необязательно, главное понять,
        что в школе не просто так говорили про параллельный перенос и совмещение фигур и тел,
        такие преобразования являются изометрией(иногда движением).\\

        Спросить про определение ориентации на примере буквы L и зеркала,
        можно также привести куб и ориентированный объем.

        \item Это, конечно, не строгое определение, так как по хорошему нужно говорить про базис,
        но понимания на конкретных примерах будет достаточно, так как эта тема нужная
        для понимания свойств фигур и их преобразований.

        \item Всякое движение является изометрией, но не наоборот.
        движением на плоскостью является параллельный перенос и поворот,
        в пространстве поворот относительно оси.\\
        Является ли движением симметрия относительно точки/прямой/плоскости?

        \item Тут более общее понятие, лучше всего обратиться к примеру.
        Путем перестановок доказать, что других преобразований быть не может,
        обсудить коммутативность и ассоциативность приведенных в примере преобразований.

        \item Теперь можно привести другой способ задачи геометрии - способ Клейна,
        спросить что будет, если отрицать некоторые постулаты Евклида.

        \item Классификация неевклидовых геометрий

        \item Конформно-евклидова модель, модель Пуанкаре
        Модель плоскости Лобачевского, предложенная Бельтрами(Бельтрами, Эудженио).
        За плоскость Лобачевского принимается внутренность круга, прямыми
        считаются дуги окружностей, перпендикулярных окружности данного круга, и
        его диаметры, движениями — преобразования, получаемые комбинациями инверсий
        (OP * OP' = R^2) относительно окружностей, дуги которых служат прямыми.


        \item Изометрия в модели Пуанкаре, движение и изометрия на сфере.

        \item Пример сферического объекта.


    \end{enumerate}


\end{document}
