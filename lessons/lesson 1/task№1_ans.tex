\documentclass[a4paper, 12pt]{article}

\usepackage[english,russian]{babel}
\usepackage[T2A]{fontenc}
\usepackage[utf8]{inputenc}
\usepackage{geometry}
\usepackage{enumitem}
\usepackage{setspace}
\usepackage{amssymb}
\usepackage{graphicx}
\usepackage{wrapfig}
\usepackage{float}
\usepackage{amsmath}
\usepackage{textcomp}
\usepackage{dsfont}

\geometry{top=5mm, left=1cm}
%\setlength{\parindent}{0}
\renewcommand{\arraystretch}{1.2}
\linespread{1}

\begin{document}
    \begin{center}
        \textbf{Сферическая геометрия №1}\\
        Входное
    \end{center}

    \textbf{№1}
    Вычислите:
    \begin{enumerate}
        \item Длину окружности радиуса $r = 12$ см.\\
        Ответ: $24\pi$

        \item Площадь сферы радиуса $R = 4$ см.\\
        Ответ: $64\pi$
        \
        \item $\sin\left(\frac{\pi}{2}\right)$\\
        Ответ: 1

        \item $\cos\left(\frac{3\pi}{4}\right)$\\
        Ответ: $\frac{\sqrt {2}}{2}$
        \item $\sin ^ 2 x + \cos ^ 2 x - 1$\\
        Ответ: 0
    \end{enumerate}\\


    \textbf{№2}
    Дан треугольник $ABC$, $AB = 3$ см, $BC = 4$ см, $AC = 5$ см.
    Найти $\cos(AB \textasciicircum BC)$.\\
    Ответ: 0(треугольник прямоугольный)\\


    \textbf{№3}
    Дан треугольник $ABC$, $AB = 3$ см, $BC = 4$ см, $AC = 6$ см.
    Найти $\cos(AB \textasciicircum BC)$.\\

    По теореме косинусов:
    \[  AC ^ 2 = AB ^ 2 + BC ^ 2 - 2 AB * BC * \cos(AB \textasciicircum BC) \]
    \[  36 = 9 + 16 - 24\cos(AB \textasciicircum BC) \]
    \[  11 = - 24\cos(AB \textasciicircum BC) \]
    \[ \cos(AB \textasciicircum BC)  = -\frac{11}{24}\]

    Ответ: $-\frac{11}{24}$\\


    \textbf{№4}
    Дан треугольник $ABC$, $AB = 3$ см, $BC = 4$ см, $\sin(C) = 0,3$.
    Найти $\sin(A)$.\\

    \textbf{Решение}\\

    По теореме синусов:
    \[
        \frac{AB}{\sin(C)} = \frac{BC}{\sin(A)}
    \]
    \[ \sin(A) = \frac{\sin(C)*BC}{AB} \]
    \[
        \frac{0,3 * 4}{3} = 0,4
    \]

    Ответ: 0,4


\end{document}
