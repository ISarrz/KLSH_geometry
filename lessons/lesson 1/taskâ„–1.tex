\documentclass[a4paper, 12pt]{article}

\usepackage[english,russian]{babel}
\usepackage[T2A]{fontenc}
\usepackage[utf8]{inputenc}
\usepackage{geometry}
\usepackage{enumitem}
\usepackage{setspace}
\usepackage{amssymb}
\usepackage{graphicx}
\usepackage{wrapfig}
\usepackage{float}
\usepackage{amsmath}
\usepackage{textcomp}
\usepackage{dsfont}

\geometry{top=5mm, left=1cm}
%\setlength{\parindent}{0}
\renewcommand{\arraystretch}{1.2}
\linespread{1}

\begin{document}
    \begin{center}
        \textbf{Сферическая геометрия №1}\\
        Сечение сферы
    \end{center}

    \begin{center}
        \textbf{№1}
    \end{center}

    Что получиться в сечении сферы радиуса $R$ плоскостью, удаленной от центра сферы на $H$, если:\\
    \begin{minipage}[t]{0.5\textwidth}
        \begin{enumerate}
            \setcounter{enumi}{0}
            \item $R = 5$ см, $H = 6$ см.
            \item $R = 2$ см, $H = 2$ см.
            \item $R = 4$ см, $H = 1$ см.
            \item $R = 4$ см, $H = \sqrt {15}$ см.
            \item $R = 5,65$ см,$H = \sqrt{32}$ см.
            \item $R = 1$ см, $H = (\sqrt{2} + 5)^{\sqrt{4}\left(\frac{2}{\sqrt{2}} - \sqrt{2}\right)}$ см.
        \end{enumerate}
    \end{minipage}
    \begin{minipage}[t]{0.5\textwidth}
        \begin{enumerate}
            \setcounter{enumi}{6}
            \item $R = 2 ^ {100}$ см, $H = 100 ^ 2$ см.
            \item $R = 200 ^ {300}$ см, $H = 300 ^ {200}$ см.
            \item $R = \sin(0,6)$ см, $H = \sin(1,8)$ см.
            \item $R = \sqrt{2 - \frac{1}{4}\sqrt{3 + \sqrt{5}}}$ см, $H = 1,2$ см.
            \item $R = \pi ^ e$ см, $H = e ^ \pi$ см.
            \item $R = 2\sqrt{18} + 2\sqrt{18}\sqrt{2}\sqrt{3} + 3\sqrt{2}$ см, \\
            $H = 2\sqrt{8} + 2\sqrt{8}\sqrt{2}\sqrt{3} + 3\sqrt{8}$ см.
        \end{enumerate}
    \end{minipage}

     \begin{center}
        \textbf{№2}
    \end{center}

    Найти радиус сферы $R$, радиус окружности сечения $r$, длину окружности $L$, расстояние $H$ от центра сферы
    до плоскости сечения, если:
    \begin{enumerate}
        \item $R = 5$ см, $r = 2$ см.
        \item $L = 6$ см, $H = 7$ см.
    \end{enumerate}
\end{document}
