\documentclass[a4paper, 12pt]{article}

\usepackage[english,russian]{babel}
\usepackage[T2A]{fontenc}
\usepackage[utf8]{inputenc}
\usepackage{geometry}
\usepackage{enumitem}
\usepackage{setspace}
\usepackage{amssymb}
\usepackage{graphicx}
\usepackage{wrapfig}
\usepackage{float}
\usepackage{amsmath}
\usepackage{textcomp}
\usepackage{dsfont}

\geometry{top=5mm, left=1cm}
%\setlength{\parindent}{0}
\renewcommand{\arraystretch}{1.2}
\linespread{1}

\begin{document}
    \begin{center}
        \textbf{Сферическая геометрия №5}\\
        Координаты.
    \end{center}

    \begin{center}
        \textbf{№ 1}
    \end{center}

    Вы бедный штурман, которому поручили вычислить направления пути в формате Юг/Север-Запад/Восток
    на протяжении всего пути, заданного координатами.
    Найдите направления, в каком полушарии лежит маршрут, задайте координаты в терминах широты и долготы, если даны следующие координаты:\\
    $56, 92$; $53, 92$; $52, 91$; $54, 95$

    \begin{center}
        \textbf{№ 2}
    \end{center}

    Жители Москвы, которая имеет координаты $55.75;$ $37.62$ решили послать
    жителям Чикаго, расположенного по координатам $41.87$; $-87.62$ подарки под новый год,
    но это желание пришло им слишком поздно, когда до нового года оставалось всего пол часа,
    поэтому самолет долететь точно не успеет, однако жителям москвы стало интересно,
    сможет ли межконтинентальная баллистическая ракета РС-28 «Сармат», имеющая максимальную дальность
    полета 18 000 км и скорость около 6 км/с доставить подарки к новому году вовремя?
    Радиус земли считайте 6400 км.

    \begin{center}
        \textbf{№ 3}
    \end{center}

    Найдите центральный угол, опирающийся на дугу, заданную двумя точками с известными координатами

    \begin{center}
        \textbf{№ 4}
    \end{center}

    Докажите, что в стереографической проекции сохраняются углы.

\end{document}
