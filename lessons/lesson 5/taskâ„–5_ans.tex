\documentclass[a4paper, 12pt]{article}

\usepackage[english,russian]{babel}
\usepackage[T2A]{fontenc}
\usepackage[utf8]{inputenc}
\usepackage{geometry}
\usepackage{enumitem}
\usepackage{setspace}
\usepackage{amssymb}
\usepackage{graphicx}
\usepackage{wrapfig}
\usepackage{float}
\usepackage{amsmath}
\usepackage{textcomp}
\usepackage{dsfont}

\geometry{top=5mm, left=1cm}
%\setlength{\parindent}{0}
\renewcommand{\arraystretch}{1.2}
\linespread{1}

\begin{document}
    \begin{center}
        \textbf{Сферическая геометрия №5}\\
        Координаты.
    \end{center}

    \begin{center}
        \textbf{№ 1}
    \end{center}

    Вы бедный штурман, которому поручили вычислить направления пути в формате Юг/Север-Запад/Восток
    на протяжении всего пути, заданного координатами.
    Найдите направления, в каком полушарии лежит маршрут, задайте координаты в терминах широты и долготы, если даны следующие координаты:\\
    $56, 92$; $53, 92$; $52, 91$; $54, 95$

    \textbf{Решение}\\

    Весь маршрут лежит в северной широте и восточной долготе.\\
    Направления: ЮГ, Юго-Запад, Северо-Восток.\\


    \begin{center}
        \textbf{№ 2}
    \end{center}

    Жители Москвы, которая имеет координаты $55.75;$ $37.62$ решили послать
    жителям Чикаго, расположенного по координатам $41.87$; $-87.62$ подарки под новый год,
    но это желание пришло им слишком поздно, когда до нового года оставалось всего пол часа,
    поэтому самолет долететь точно не успеет, однако жителям москвы стало интересно,
    сможет ли межконтинентальная баллистическая ракета РС-28 «Сармат», имеющая максимальную дальность
    полета 18 000 км и скорость около 6 км/с доставить подарки к новому году вовремя?
    Радиус земли считайте 6400 км.

    \textbf{Решение}\\

    \[
        \phi_1 = 55.75^{\circ}, \lambda_1 = 37.62^{\circ}
    \]
    \[
        \phi_2 = 41.87^{\circ}, \lambda_2 = -87.62^{\circ}
    \]
    \[
        \phi_1 = 0.973, \phi_2 = 0.731
    \]
    \[
        \lambda_1 = 0.657, \lambda_2 = -1.529
    \]
    \[
        \cos \frac{d}{R} = \sin \phi_1 \sin \phi_2 + \cos\phi_1\cos\phi_2 \cos\delta\lambda =
        0.311
    \]
    \[
        \frac{d}{R} = 1.256
    \]
    \[
        d = 8040 \text{ км}
    \]
    \[
        \frac{8040}{6} = 1340 \text{ секунд} = 22.3 \text{ минут}
    \]

    Ответ: межконтинентальная баллистическая ракета РС‑28 «Сармат» с дальностью
    18000 км и скоростью около 6 км/с способна доставить подарки к Новому году.

    \begin{center}
        \textbf{№ 3}
    \end{center}

    Найдите центральный угол, опирающийся на дугу, заданную двумя точками с известными координатами

    \textbf{Решение}\\

    \[
        \arccos( \sin \phi_1 \sin \phi_2 + \cos\phi_1\cos\phi_2 \cos\delta\lambda)
    \]

    \begin{center}
        \textbf{№ 4}
    \end{center}

    Докажите, что в стереографической проекции сохраняются углы.

    \textbf{Решение}\\

    Рассматриваем углы между касательных, показываем, что прямая проекции проходит
    через касательные.
\end{document}
