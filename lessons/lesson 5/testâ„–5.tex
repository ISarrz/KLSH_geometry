\documentclass[a4paper, 12pt]{article}

\usepackage[english,russian]{babel}
\usepackage[T2A]{fontenc}
\usepackage[utf8]{inputenc}
\usepackage{geometry}
\usepackage{enumitem}
\usepackage{setspace}
\usepackage{amssymb}
\usepackage{graphicx}
\usepackage{wrapfig}
\usepackage{float}
\usepackage{amsmath}
\usepackage{textcomp}
\usepackage{dsfont}

\geometry{top=5mm, left=1cm}
%\setlength{\parindent}{0}
\renewcommand{\arraystretch}{1.2}
\linespread{1}

\begin{document}
    \begin{center}
        \textbf{Сферическая геометрия тест №5}\\
        Площадь двуугольника, площадь треугольника.
    \end{center}

    \textbf{ФИО:}

    \begin{center}
        \textbf{№ 1}
    \end{center}

     Две сферические прямые пересекаются под углом $\frac{\pi}{2}$.
    Найдите чему равны площади каждого двуугольника, образованного этими прямыми, и посчитайте их сумму,
    если радиус сферы $R=52$ см.

    \begin{center}
        \textbf{№ 2}
    \end{center}

    На сфере дан треугольник, два угла которого равны $\frac{\pi}{2}$, а третий равен $\frac{\pi}{3}$.
    Из вершины третьего угла на противолежащую сторону опустили медиану.
    Найдите площади получившихся треугольников, если радиус сферы равен $R$.

\end{document}
