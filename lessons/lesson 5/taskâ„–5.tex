\documentclass[a4paper, 12pt]{article}

\usepackage[english,russian]{babel}
\usepackage[T2A]{fontenc}
\usepackage[utf8]{inputenc}
\usepackage{geometry}
\usepackage{enumitem}
\usepackage{setspace}
\usepackage{amssymb}
\usepackage{graphicx}
\usepackage{wrapfig}
\usepackage{float}
\usepackage{amsmath}
\usepackage{textcomp}
\usepackage{dsfont}

\geometry{top=5mm, left=1cm}
%\setlength{\parindent}{0}
\renewcommand{\arraystretch}{1.2}
\linespread{1}

\begin{document}
    \begin{center}
        \textbf{Сферическая геометрия №5}\\
        Координаты.
    \end{center}


    \textbf{№ 1}
    Вы бедный штурман, которому поручили вычислить направления пути в формате Юг/Север-Запад/Восток
    на протяжении всего пути, заданного координатами.
    Найдите направления, в каком полушарии лежит маршрут, задайте координаты в терминах широты и долготы, если даны следующие координаты:\\
    \[
        56;92 \quad 53;92 \quad 52;91 \quad 54;95
    \]\\


    \textbf{№ 2}
    Компания «Монофон», главный офис которой расположен в Москве(55.75; 37.62)
    решили поздравить своих коллег из Чикаго(41.87; -87.62), отправив им открытку.
    К сожалению, решение было принято слишком поздно, до нового года оставалось всего
    пол часа.
    Среди сотрудников этой кампании работает молодой программист Артемий, который
    увлекается прыжками в длину, ходят слухи, что он может прыгнуть на 18 000 км,
    причем скорость в полете будет постоянна - 6 км/с.

    Проверьте, есть ли надежда, что открытка будет доставлена вовремя?
    Радиус земли считайте 6400 км.\\


    \textbf{№ 3}
    Докажите, что в стереографической проекции сохраняются углы.

\end{document}
