\documentclass[a4paper, 12pt]{article}

\usepackage[english,russian]{babel}
\usepackage[T2A]{fontenc}
\usepackage[utf8]{inputenc}
\usepackage{geometry}
\usepackage{enumitem}
\usepackage{setspace}
\usepackage{amssymb}
\usepackage{graphicx}
\usepackage{wrapfig}
\usepackage{float}
\usepackage{amsmath}
\usepackage{textcomp}
\usepackage{dsfont}

\geometry{top=5mm, left=1cm}
%\setlength{\parindent}{0}
\renewcommand{\arraystretch}{1.2}
\linespread{1}

\begin{document}
    \begin{center}
        \textbf{Сферическая геометрия №4}\\
        Фигуры.
    \end{center}

    \begin{center}
        \textbf{№ 1}
    \end{center}
    Пусть $n$ - число углов в фигуре, какое минимальное $n$ возможно на евклидовой плоскости?
    А на сферической?
    Дайте ее определение, найдите площадь такой сферической фигуры, если для нее известны все углы.

    \begin{center}
        \textbf{№ 2}
    \end{center}

    Дайте определение сферического треугольника, его сторон, углов, вершин.
    Найдите его площадь.

    \begin{center}
        \textbf{№ 3}
    \end{center}

    Сравните сумму углов треугольника на сфере и на евклидовой плоскости.
    Выведите формулу, вычисляющую сферический «дефект».
    Возможен ли треугольник, у которого все углы $90^\circ$,
    а треугольник все стороны которого лежат на экваторе?
    Найдите их стороны и площади.

    \begin{center}
        \textbf{№ 4}
    \end{center}

    Выведите формулу для площади сферического $n$-угольника и формулу сферического дефекта для него.

    \begin{center}
        \textbf{№ 5}
    \end{center}
    На сфере дан треугольник $\triangle ABC$, на уго сторонах $AB$ и $AC$ взяли
    точки $P$ и $Q$ так, что $\angle APQ = \angle ABC$ и $\angle AQP = \angle ACB$.
    Найдите площадь $PQCB$.

    \begin{center}
        \textbf{№ 6}
    \end{center}
    Проверьте, всегда ли медианы сферического треугольника пересекаются в одной точке?
    А высоты и биссектрисы?

    \begin{center}
        \textbf{№ 7}
    \end{center}
    Когда для сферического треугольника существует описанная окружность?

    \begin{center}
        \textbf{№ 8}
    \end{center}
    Две сферические прямые пересекаются под углом $\frac{\pi}{6}$.
    Найдите чему равны площади каждого двуугольника, образованного этими прямыми, и посчитайте их сумму,
    если радиус сферы $R=12$ см.

    \begin{center}
        \textbf{№ 9}
    \end{center}

    Две сферические прямые пересекаются под углом $\alpha$, третья прямая пересекает две проведенных прямых
    под одинаковыми углами.
    Найдите эти углы, если радиус сферы равен $R$, а площадь сферического треугольника, образованного этими прямыми $S$.

    \begin{center}
        \textbf{№ 10}
    \end{center}

    Чему равна площадь сферического треугольника, образованного полюсом и двумя сопряженными с ним точками,
    если сферическое расстояние между этими точками равно $h$, а радиус сферы равен $R$.

    \begin{center}
        \textbf{№ 11}
    \end{center}

    Дан сферический треугольник с площадью $S$.
    Найдите площадь треугольника с такими же углами на сфере с радиусом в два раза больше.

    \begin{center}
        \textbf{№ 12}
    \end{center}

    Два диаметра, соединяющих пары полюсов пересекаются под углом $\frac{\pi}{6}$,
    чему равны площади двуугольников, образованных их полярами, если радиус сферы равен $19$?

    \begin{center}
        \textbf{№ 13}
    \end{center}

    На сфере даны два равнобедренных треугольника, имеющих один равный угол.
    Отношение углов при основании первого треугольника ко второму равно $\delta$.
    Найдите отношение площадей этих треугольников

    \begin{center}
        \textbf{№ 14}
    \end{center}

    На сфере дан треугольник, все углы которого равны $90^\circ$.
    На одну из сторон опустили медиану.
    Найдите чему равна площадь получившихся треугольников, если площадь изначального треугольника равна $S$.
\end{document}
