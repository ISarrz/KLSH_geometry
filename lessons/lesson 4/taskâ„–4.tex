\documentclass[a4paper, 12pt]{article}

\usepackage[english,russian]{babel}
\usepackage[T2A]{fontenc}
\usepackage[utf8]{inputenc}
\usepackage{geometry}
\usepackage{enumitem}
\usepackage{setspace}
\usepackage{amssymb}
\usepackage{graphicx}
\usepackage{wrapfig}
\usepackage{float}
\usepackage{amsmath}
\usepackage{textcomp}
\usepackage{dsfont}

\geometry{top=5mm, left=1cm}
%\setlength{\parindent}{0}
\renewcommand{\arraystretch}{1.2}
\linespread{1}

\begin{document}
    \begin{center}
        \textbf{Сферическая геометрия №4}\\
        Фигуры.
    \end{center}

    \textbf{№ 1}
    Пусть $n$ - число углов в фигуре, какое минимальное $n$ возможно на евклидовой плоскости?
    А на сферической?
    Дайте определение такой фигуры, найдите площадь такой сферической фигуры, если для нее известны все углы.\\


    \textbf{№ 2}
    Дайте определение сферического треугольника, его сторон, углов, вершин.
    Найдите его площадь(см. №1).\\


    \textbf{№ 3}
    Сравните сумму углов треугольника на сфере и на евклидовой плоскости.
    Выведите формулу, вычисляющую сферический «дефект».
    Возможен ли треугольник, у которого все углы $90^\circ$,
    а треугольник все стороны которого лежат на экваторе?
    Найдите их стороны и площади.\\


    \textbf{№ 4}
    Выведите формулу для площади сферического $n$-угольника и формулу сферического дефекта для него.\\


    \textbf{№ 5}
    На сфере дан треугольник $\triangle ABC$, на его сторонах $AB$ и $AC$ взяли
    точки $P$ и $Q$ так, что $\angle APQ = \angle ABC$ и $\angle AQP = \angle ACB$.
    Найдите площадь $PQCB$.\\


    \textbf{№ 6}
    Проверьте, всегда ли медианы сферического треугольника пересекаются в одной точке?
    А высоты и биссектрисы?\\


    \textbf{№ 7}
    Когда для сферического треугольника существует описанная окружность?\\


    \textbf{№ 10}
    Чему равна площадь сферического треугольника, образованного полюсом и двумя сопряженными с ним точками,
    если сферическое расстояние между этими точками равно $h$, а радиус сферы равен $R$.\\


     \textbf{№ 11}
    Дан сферический треугольник с площадью $S$.
    Найдите площадь треугольника с такими же углами на сфере с вдвое большим радиусом.\\


    \textbf{№ 12}
    Два диаметра, соединяющих пары полюсов пересекаются под углом $\alpha$,
    чему равны площади двуугольников, образованных их полярами, если радиус сферы равен $R$?\\


    \textbf{№ 13}
    На сфере даны два равнобедренных треугольника, имеющих один равный угол.
    Отношение углов при основании первого треугольника ко второму равно $\delta$.
    Найдите отношение площадей этих треугольников.\\


    \textbf{№ 14}
    На сфере дан треугольник, все углы которого равны $90^\circ$.
    На одну из сторон опустили медиану.
    Найдите чему равна площадь получившихся треугольников, если площадь изначального треугольника равна $S$.
\end{document}
