\documentclass[a4paper, 12pt]{article}

\usepackage[english,russian]{babel}
\usepackage[T2A]{fontenc}
\usepackage[utf8]{inputenc}
\usepackage{geometry}
\usepackage{enumitem}
\usepackage{setspace}
\usepackage{amssymb}
\usepackage{graphicx}
\usepackage{wrapfig}
\usepackage{float}
\usepackage{amsmath}
\usepackage{textcomp}
\usepackage{dsfont}

\geometry{top=5mm, left=1cm}
%\setlength{\parindent}{0}
\renewcommand{\arraystretch}{1.2}
\linespread{1}

\begin{document}
    \begin{center}
        \textbf{Сферическая геометрия тест №2}\\
        Сечения сферы
    \end{center}

    \textbf{ФИО:}

    \begin{center}
        \textbf{№ 1}
    \end{center}

    Что получится в сечении сферы радиуса $R$ плоскостью, удаленной от центра сферы на $h$, если:
    \begin{enumerate}
        \item $R = 239$ см, $h = 239,1$ см;
        \item $R = 0,1$ см, $h = 0,01$ см;
        \item $R = \pi$ см, $h = \pi$ см;
        \item $R = 2\sqrt{5}$ см, $h = \sqrt{3}$ см
    \end{enumerate}

    \begin{center}
        \textbf{№ 2}
    \end{center}

    Назовите виды окружностей, которые получаются при сечении сферы плоскостью.
    \\ \ \\ \ \\

    \begin{center}
        \textbf{№ 3}
    \end{center}

     Расстояние от центра сферы радиуса 13 см до секущей плоскости равно 5 см.
    Вычислите радиус и длину полученной в сечении окружности.

\end{document}
