\documentclass[a4paper, 12pt]{article}

\usepackage[english,russian]{babel}
\usepackage[T2A]{fontenc}
\usepackage[utf8]{inputenc}
\usepackage{geometry}
\usepackage{enumitem}
\usepackage{setspace}
\usepackage{amssymb}
\usepackage{graphicx}
\usepackage{wrapfig}
\usepackage{float}
\usepackage{amsmath}
\usepackage{textcomp}
\usepackage{dsfont}

\geometry{top=5mm, left=1cm}
%\setlength{\parindent}{0}
\renewcommand{\arraystretch}{1.2}
\linespread{1}

\begin{document}
    \begin{center}
        \textbf{Сферическая геометрия №2}\\
        Прямые, полюсы, поляры
    \end{center}

    \begin{center}
        \textbf{№1}
    \end{center}

    Докажите, что если плоскость проходит через центр сферы, то она содержит ее диаметр.
    \begin{center}
        \textbf{№2}
    \end{center}

    Докажите, что прямой на сфере соответствует единственная пара полюсов.

    \begin{center}
        \textbf{№3}
    \end{center}

    Докажите, что двум диаметрально противоположным точкам на сфере соответствует единственная поляра.

    \begin{center}
        \textbf{№4}
    \end{center}

    Докажите, что cуществует единственная прямая, проходящая через две данные различные точки,
    кроме случая, когда эти точки диаметрально противоположны;
    тогда таких прямых бесконечно много.

    \begin{center}
        \textbf{№5}
    \end{center}

    Точки $A$ и $C$ - полярно сопряженные на окружности радиуса $R$.
    Найдите Евклидово расстояние между этими точками.

    \begin{center}
        \textbf{№6}
    \end{center}

    Евклидово расстояние между двумя полярно сопряженными точками равно 12 см.
    Чему равна длина прямой на сфере?

    \begin{center}
        \textbf{№7}
    \end{center}

    Угол между двумя секущими плоскостями равен $\alpha$.
    Чему равен угол между двумя прямыми,
    каждая из которых соединяет полюсы соответсвующий плоскостей?

    \begin{center}
        \textbf{№8}
    \end{center}

    При каком условии две прямые на сфере содержат полюсы друг друга?

    \begin{center}
        \textbf{№9}
    \end{center}

    Угол между секущими плоскостями, проходящими через центр сферы равен $\alpha$, радиус сферы равен $R$.
    Найдите Евклидово расстояние между всеми полюсами.

    \begin{center}
        \textbf{№10}
    \end{center}

    Чему равна площадь треугольника, образованного двумя полисами и полярно сопряженной с ними точкой,
    если радиус сферы $R$.

    \begin{center}
        \textbf{№11}
    \end{center}

    Угол между двумя секущими плоскостями равен $\alpha$.
    Чему равен угол между диаметром, соединяющим одну пару полюсов одной плоскости, и другой плоскостью?


    \begin{center}
        \textbf{№12}
    \end{center}

    Длина большой окружности равно l, радиус сферы R.
    На большой окружности выбраны две точки, расстояние между которыми h.
    Найдите периметр и площадь треугольника,
    составленного из этих точек окружности и полярно сопряженной с ними точкой сферы.


\end{document}
