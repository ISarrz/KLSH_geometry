\documentclass[a4paper, 12pt]{article}

\usepackage[english,russian]{babel}
\usepackage[T2A]{fontenc}
\usepackage[utf8]{inputenc}
\usepackage{geometry}
\usepackage{enumitem}
\usepackage{setspace}
\usepackage{amssymb}
\usepackage{graphicx}
\usepackage{wrapfig}
\usepackage{float}
\usepackage{amsmath}
\usepackage{textcomp}
\usepackage{dsfont}

\geometry{top=5mm, left=1cm}
%\setlength{\parindent}{0}
\renewcommand{\arraystretch}{1.2}
\linespread{1}

\begin{document}
    \begin{center}
        \textbf{Сферическая геометрия №2}\\
        Сечения сферы
    \end{center}

    \textbf{№1} Расстояние от центра сферы радиуса R до секущей плоскости равно $d$.
    Вычислите радиус окружности, полученной в сечении плоскостью.\\


    \textbf{№2} Секущая плоскость проходит через конец диаметра сферы радиуса $R$ так,
    что угол между диаметром и плоскостью равен $\alpha$.
    Найдите длину окружности, получившейся в сечении.\\


    \textbf{№3} Покажите, что если плоскость проходит через центр сферы, то она содержит ее диаметр.\\


    \textbf{№4} Покажите, что сферической прямой соответствует единственная пара полюсов.\\


    \textbf{№5} Покажите, что двум диаметрально противоположным точкам на сфере соответствует единственная поляра.\\


    \textbf{№6} Покажите, что cуществует единственная сферическая прямая, проходящая через две данные различные точки,
    кроме случая, когда эти точки диаметрально противоположны;
    тогда таких прямых бесконечно много.\\


    \textbf{№7} Постройте параллельные сферические прямые или докажите, что это невозможно.\\


    \textbf{№8} Точки $A$ и $C$ - полярно сопряженные на окружности радиуса $R$.
    Найдите евклидово расстояние между этими точками.\\


    \textbf{№9} Когда поляры(плоскости) трех точек пересекаются по прямой?\\


    \textbf{№10} При каком условии две прямые на сфере содержат полюсы друг друга?\\


    \textbf{№11} Чему равна площадь евклидового треугольника, образованного двумя полисами и полярно сопряженной с ними точкой,
    если радиус сферы $R$.\\



\end{document}
