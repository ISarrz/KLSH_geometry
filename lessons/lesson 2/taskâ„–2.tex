\documentclass[a4paper, 12pt]{article}

\usepackage[english,russian]{babel}
\usepackage[T2A]{fontenc}
\usepackage[utf8]{inputenc}
\usepackage{geometry}
\usepackage{enumitem}
\usepackage{setspace}
\usepackage{amssymb}
\usepackage{graphicx}
\usepackage{wrapfig}
\usepackage{float}
\usepackage{amsmath}
\usepackage{textcomp}
\usepackage{dsfont}

\geometry{top=5mm, left=1cm}
%\setlength{\parindent}{0}
\renewcommand{\arraystretch}{1.2}
\linespread{1}

\begin{document}
    \begin{center}
        \textbf{Сферическая геометрия №2}\\
        Сечения сферы
    \end{center}

    \textbf{№1} Что получиться в сечении сферы радиуса $R$ плоскостью, удаленной от центра сферы на $H$, если:\\

    \begin{enumerate}
        \setcounter{enumi}{0}
        \item $R = 5$ см, $H = 6$ см.

        \item $R = 2$ см, $H = 2$ см.

        \item $R = 4$ см, $H = 1$ см.

    \end{enumerate}\\

    \textbf{№2} Расстояние от центра сферы радиуса R до секущей плоскости равно $d$.
    Вычислите:
    \begin{enumerate}
        \item Радиус окружности, полученной в сечении плоскостью, если $R = 5$ см, $d = 3$ см.
        \item Длину окружности, полученной в сечении плоскостью, если $R = 12$ см, $d = 8$ см.
    \end{enumerate}\\

    \textbf{№3} Секущая плоскость проходит через конец диаметра сферы радиуса $R$ так,
    что угол между диаметром и плоскостью равен $\alpha$.
    Найдите длину окружности, получившейся в сечении, если:
    \begin{enumerate}
        \item $R = 2$ см, $\alpha = 30 ^{\circ}$;
        \item $R = 5$ см, $\alpha = 45 ^{\circ}$
    \end{enumerate}\\

    \textbf{№4} Докажите, что если плоскость проходит через центр сферы, то она содержит ее диаметр.\\

    \textbf{№5} Докажите, что прямой на сфере соответствует единственная пара полюсов.\\

    \textbf{№6} Докажите, что двум диаметрально противоположным точкам на сфере соответствует единственная поляра.\\

    \textbf{№7} Докажите, что cуществует единственная прямая, проходящая через две данные различные точки,
    кроме случая, когда эти точки диаметрально противоположны;
    тогда таких прямых бесконечно много.\\

    \textbf{№8} Постройте параллельные прямые на сфере или докажите, что это невозможно.\\

    \textbf{№9} Точки $A$ и $C$ - полярно сопряженные на окружности радиуса $R$.
    Найдите Евклидово расстояние между этими точками.\\

    \textbf{№10} Угол между двумя секущими плоскостями, проходящими через центр сферы, равен $\alpha$.
    Чему равен угол между двумя прямыми,
    каждая из которых соединяет полюсы соответсвующий плоскостей?\\

    \textbf{№11} Когда поляры(плоскости) трех точек пересекаются по прямой?\\

    \textbf{№12} При каком условии две прямые на сфере содержат полюсы друг друга?\\

    \textbf{№13} Чему равна площадь треугольника, образованного двумя полисами и полярно сопряженной с ними точкой,
    если радиус сферы $R$.\\

    \textbf{№14} Угол между двумя секущими плоскостями равен $\alpha$.
    Чему равен угол между диаметром, соединяющим одну пару полюсов одной плоскости, и другой плоскостью?\\

\end{document}
