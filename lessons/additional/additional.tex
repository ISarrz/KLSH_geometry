\documentclass[a4paper, 12pt]{article}

\usepackage[english,russian]{babel}
\usepackage[T2A]{fontenc}
\usepackage[utf8]{inputenc}
\usepackage{geometry}
\usepackage{enumitem}
\usepackage{setspace}
\usepackage{amssymb}
\usepackage{graphicx}
\usepackage{wrapfig}
\usepackage{float}
\usepackage{amsmath}
\usepackage{textcomp}
\usepackage{dsfont}

\geometry{top=5mm, left=1cm}
%\setlength{\parindent}{0}
\renewcommand{\arraystretch}{1.2}
\linespread{1}

\begin{document}
    \begin{center}
        \textbf{Сферическая геометрия задачи}\\
        Дополнительные задачи
    \end{center}

    \begin{center}
       \textbf{Геометрии}
    \end{center}

    \textbf{№1}
    Приведите все изометрии(группу симметрий) единичного квадрата.\\


    \begin{center}
        \textbf{Сечения сферы}
    \end{center}

    \textbf{№2}
    Угол между двумя секущими плоскостями, проходящими через центр сферы, равен $\alpha$.
    Чему равен угол между двумя прямыми,
    каждая из которых соединяет полюсы соответсвующий плоскостей?\\


    \begin{center}
        \textbf{Прямые, полюсы, поляры}
    \end{center}

    \textbf{№ 3}
    Угол между двумя секущими плоскостями равен $\alpha$.
    Чему равен угол между двумя прямыми,
    каждая из которых соединяет полюсы соответсвующий плоскостей?\\


     \textbf{№4} Угол между двумя секущими плоскостями равен $\alpha$.
    Чему равен угол между диаметром, соединяющим одну пару полюсов одной плоскости, и другой плоскостью?\\


    \textbf{№ 5}
    Чему равна площадь евклидового четырехугольника, образованного двумя полисами и диаметрально противоположными точками поляры,
    если радиус сферы $R$.\\


    \begin{center}
        \textbf{Расстояние между точками, углы между прямыми, сферические окружности}
    \end{center}

    \begin{center}
        \textbf{№ 1}
    \end{center}

    На сфере радиуса 7 см построена сферическая окружность радиусом 3 см.
    Чему равен радиус малой окружности, совпадающей с данной?

    \begin{center}
        \textbf{№ 2}
    \end{center}

    Проведены две сферические прямые, пересекающиеся под углом $\alpha$,
    перпендикулярно к ним проведена третья сферическая прямая.
    Чему равен угол $\alpha$, если сферическое расстояние между точками пересечения
    двух прямых третьей равно $h$, а радиус сферы равен $R$?

    \begin{center}
        \textbf{Фигуры}
    \end{center}

    \begin{center}
        \textbf{№ 8}
    \end{center}
    Две сферические прямые пересекаются под углом $\frac{\pi}{6}$.
    Найдите чему равны площади каждого двуугольника, образованного этими прямыми, и посчитайте их сумму,
    если радиус сферы $R=12$ см.

    \begin{center}
        \textbf{№ 9}
    \end{center}

    Две сферические прямые пересекаются под углом $\alpha$, третья прямая пересекает две проведенные прямые
    под одинаковыми углами.
    Найдите эти углы, если радиус сферы равен $R$, а площадь сферического треугольника, образованного этими прямыми $S$.

    \begin{center}
        \textbf{№ 1}
    \end{center}

     Две сферические прямые пересекаются под углом $\frac{\pi}{2}$.
    Найдите чему равны площади каждого двуугольника, образованного этими прямыми, и посчитайте их сумму,
    если радиус сферы $R=52$ см.

    \begin{center}
        \textbf{№ 2}
    \end{center}

    На сфере дан треугольник, два угла которого равны $\frac{\pi}{2}$, а третий равен $\frac{\pi}{3}$.
    Из вершины третьего угла на противолежащую сторону опустили медиану.
    Найдите площади получившихся треугольников, если радиус сферы равен $R$.
\end{document}
