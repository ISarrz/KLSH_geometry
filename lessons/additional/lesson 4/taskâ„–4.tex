\documentclass[a4paper, 12pt]{article}

\usepackage[english,russian]{babel}
\usepackage[T2A]{fontenc}
\usepackage[utf8]{inputenc}
\usepackage{geometry}
\usepackage{enumitem}
\usepackage{setspace}
\usepackage{amssymb}
\usepackage{graphicx}
\usepackage{wrapfig}
\usepackage{float}
\usepackage{amsmath}
\usepackage{textcomp}
\usepackage{dsfont}

\geometry{top=5mm, left=1cm}
%\setlength{\parindent}{0}
\renewcommand{\arraystretch}{1.2}
\linespread{1}

\begin{document}
    \begin{center}
        \textbf{Сферическая геометрия №4}\\
        Площадь двуугольника, площадь треугольника.
    \end{center}

    \begin{center}
        \textbf{№ 1}
    \end{center}

    Две сферические прямые пересекаются под углом $\frac{\pi}{6}$.
    Найдите чему равны площади каждого двуугольника, образованного этими прямыми, и посчитайте их сумму,
    если радиус сферы $R=12$ см.

    \begin{center}
        \textbf{№ 2}
    \end{center}

    Две сферические прямые пересекаются под углом $\alpha$, третья прямая пересекает две проведенных прямых
    под одинаковыми углами.
    Найдите эти углы, если радиус сферы равен $R$, а площадь сферического треугольника, образованного этими прямыми $S$.

    \begin{center}
        \textbf{№ 3}
    \end{center}

    Чему равна площадь сферического треугольника, образованного полюсом и двумя сопряженными с ним точками,
    если сферическое расстояние между этими точками равно $h$, а радиус сферы равен $R$.

    \begin{center}
        \textbf{№ 4}
    \end{center}

    Дан сферический треугольник с площадью $S$ на сфере радиуса $R$.
    Найдите площадь треугольника с такими же углами на сфере радиуса $2R$.

    \begin{center}
        \textbf{№ 5}
    \end{center}

    Два диаметра, соединяющих пары полюсов пересекаются под углом $\frac{\pi}{6}$,
    чему равны площади двуугольников, образованных их полярами, если радиус сферы равен $19$?

    \begin{center}
        \textbf{№ 6}
    \end{center}

    Может ли на сфере быть построен сферический треугольник, все углы которого $90^\circ$.
    Если такой треугольник существует, то найдите его стороны и площадь.

    \begin{center}
        \textbf{№ 7}
    \end{center}

    Пусть стороны сферического треугольника равны $a, b, c$, радиус сферы равен $R$.
    Найдите отношение площадей сферического и планиметрического треугольников, имеющих общие вершины.

    \begin{center}
        \textbf{№ 9}
    \end{center}

    На сфере радиуса $R$ построен сферический треугольник с сторонами $a, b, c$.
    Найдите евклидово расстояние между каждой парой его вершин и радиус описанной окружности
    около планиметрического треугольника, построенного на вершинах сферического.

    \begin{center}
        \textbf{№ 10}
    \end{center}

    На сфере даны два равнобедренных треугольника, имеющих один равный угол.
    Отношение углов при основании двух треугольников равно $\delta$.
    Найдите отношение площадей этих треугольников

    \begin{center}
        \textbf{№ 11}
    \end{center}

    На сфере дан треугольник, все углы которого равны $90^\circ$.
    На одну из сторон опустили медиану.
    Найдите чему равна площадь получившихся треугольников, если площадь изначального треугольника равна $S$.

\end{document}
